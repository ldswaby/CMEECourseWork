\documentclass[11pt]{article}
\usepackage{graphicx}
\usepackage{amsmath}
\usepackage[utf8]{inputenc}
\usepackage{setspace}
\usepackage{caption}
\usepackage{subcaption}
\usepackage{wrapfig}
\usepackage{caption}
\captionsetup[figure]{font=footnotesize}
\usepackage{lineno}
\usepackage{harvard}
\usepackage{parskip}
\usepackage{indentfirst}
\usepackage[a4paper,width=150mm,top=20mm,bottom=20mm,bindingoffset=6mm]{geometry}
\usepackage{textcomp}
\newcommand{\textapprox}{\raisebox{0.5ex}{\texttildelow}}

\usepackage{authblk}
\usepackage{hyperref}
\usepackage{natbib}
\bibliographystyle{abbrvnat}
\setcitestyle{authoryear,open={(},close={)}}


\title{CMEE Seminar Diary 2020-2021}
\author{Luke Swaby\\MSc. CMEE\\Faculty of Natural Sciences}

\date{2nd July 2020}

\begin{document}

    \maketitle

\newpage

\section{Sleeping in a society of honey bees: The tale of a sleep-deprived dancer and her unwitting followers, and the search for insect dreams}
\textit{Barrett Klein\\University of Wisconsin\\15/10/2020}
\\
\\
Arthropods constitute the largest phylum in the animal kingdom, yet are the least studied in the context of sleep. It has been suggested by some that sleep plays an important role in memory and language consolidation \citep{fenn2003consolidation}, and that its deprivation can therefore inhibit communication \citep{morris1960misperception, Boyce}. It is known that foragers of the western honey bee \emph{(Apis mellifera)} exhibit figure-eight-shaped waggle dances in order to inform their nest mates of food sources.  Upon reviewing the physiological indicators of sleep in bees, \cite{klein2010sleep} investigated the effect of sleep deprivation on the communicative accuracy of this famous ritual, and how a sleep-deprived bee’s impaired dances can influence follower behaviour. To this end, they built an 'insominator'—an observation hive with a thin sheet of plastic covering each side, over which they ran a pair of rare earth neodymium magnets to lightly disturb a select few individual bees, marked with yellow paint, in order to keep them awake. They found that sleep-restricted bees danced for roughly the same duration of time but at less consistent angles, indicating that while the communication of distance information was fairly unaffected, the communication of location information was compromised, leading many of the followers of these less precise sleep-deprived dancers to switch to another dancer before deciding to leave the next in search of the item of interest.

\section{Floating data points: poo and the pandemic in Scotland}
\textit{Alex Corbishley\\University of Edinburgh\\05/11/2020}
\\
\\Wastewater sampling has historically been used to ascertain the prevalence of certain biological agents in a given area, and has in recent years been proposed as a potential surveillance tool for pharmacological agents, illicit drugs and antimicrobial resistance genes. After reports emerged from Wuhan in 2020 that RNA of the novel SARS-CoV-2 coronavirus could be detected in human stool, concern was raised that this may constitute a  transmission route for the virus. However, in light of there being no evidence of faecal transmission to date, a number of countries instead seized the opportunity to attempt tracking community infection through their respective municipal wastewater networks. The idea was that if the public are shedding Sars-CoV-2 in their faeces, then samples taken from local wastewater treatment facilities could be used to concentrate and isolate the virus and then, with some understanding of the shedding rate and some other parameters relative to the contributing population, to derive a proxy for how much of the virus is in the community. It was to this end and using this methodology that the Scottish wastewater monitoring programme was established, covering approximately half of the population of Scotland (~2.6m). From 538 samples collected from 28 sites, preliminary findings indicated a nationwide positive correlation between test positivity and the numbers of both the SARS-CoV-2 RNA N1 and E gene copies found in the stool, with spikes in all three preceding spikes in deaths by a week or so. Whilst these findings can't be used to accurately predict what will happen at any point in the future, they do provide some insight into what has happened in the immediate past, and can be used as a population monitoring tool to aid in the efficient allocation of limited testing resources.

\section{Eco-evolutionary dynamics of coronaviruses}
\textit{Jessica Metcalf\\Princeton University\\06/05/2021}
\\
\\
Pandemics have shaped human history since as early as 2AD, but have been highlighted as an especially worrisome threat in recent years by the increased connectivity of the world, which enables pathogens to spread further, faster. In late 2019, a coronavirus spilled over from a bat reservoir into human populations—possibly through some intermediaries—and caused the emergence of SARS-CoV-2 in China, which quickly spread to the rest of the world via international air travel. Despite the prevalence of coronaviruses in modern day-to-day parlance, there are interestingly only four that are endemic to humans, with only three emerging in the last two decades: SARS-CoV, MERS-CoV, and SARS-CoV-2. Knowing that there exists a vast zoonotic reservoir of coronaviruses and that spillover events occur regularly, why is it that we only see four of these viruses endemic to humans? There are a number of possible explanations, many to to with the recent acceleration of globalisation and anthropogenic disruptions to natural systems, but the answer is more likely to do with limitations imposed by pathogen biology (e.g. virulence and transmissibility) and host immune space (e.g. host's CpG ratio relative to that of the virus) making evading immunity in the human population a particularly difficult task.\\
So what do we do with this information? Huge advances in our tools for measuring the landscape of immunity can inform some future developments: First, the recent and worrying suggestion that vaccines which reduce symptom severity but not infection or transmission can drive selection for higher transmissiblity and virulence if they fail to provide URT protection necessitates close monitoring of these properties. Additionally, the potential for localised antigenic adaptation in a region with little immune protection to rapidly spread elsewhere and undermine the work of vaccination efforts hitherto demands a wide and equitable distribution of vaccines as well as a long term plan to ensure two doses to as many individuals as possible, even if only a one dose strategy is implemented in the short run.



\section{Living together: cooperation in the sociable weaver}
\textit{Rita Covas\\University of Porto\\27/05/2021}
\\
\\Social cooperation is key to survival for many species in the natural world. However, it makes little sense through the lens of classical, organism-centred natural selection, which would presumably favour selfishness over altruistic behaviour. Kin selection — a phenomenon whereby individuals altruistically help to improve the reproductive success of their genetic relatives — constitutes a partial solution to this problem. However, many species display complex social cooperation across multiple levels of social organisation beyond the familial, including interspecifically. One such species is the Sociable Weaver. An easily trackable and highly social species often living in colonies integrating kin and non-kin, these birds are exemplary for studies on the origins and maintenance of social cooperation.
Through a series of experiments, \cite{covas2008helpers} were able to investigate the causes and consequences of cooperative breeding as well as the role of environmental variation in Kalahari Sociable Weavers.\\First, a food supplementation experiment (whereby half the colony was supplemented with food whilst the other was left as a control over 2 breeding seasons) resulted in a significant increase in breeders and decrease in group size (no. of helpers) in the food-supplemented group, suggesting food restrictions may be a potential reason for this behaviour. The disadvantages of helping were investigated by measuring oxidative stress levels in the population and increasing the costs of helping by clipping two primary feathers on a few young birds. The results suggested: a) a trade-off exists between helping and self-maintenance; b)  individuals adjust their behaviour to minimise costs; and c) oxidative costs influence helping decisions. As for the benefits, it was observed that pairs without helpers produced chicks with slightly lower mass in periods of low rainfall, and that fledgling success increased in proportion to colony size with helpers present, but decreased without. Similar results have been observed for other cooperative species, such as the mongoose, suggesting that the primary role of helpers in these groups is to help mitigate the adverse effects of environmental unpredictability.


\section{Tracing the evolutionary history of viruses from the Stone Age to the present using ancient DNA}
\textit{Martin Sikora\\Lundbeck Foundation GeoGenetics Centre\\10/06/2021}
\\
\\Recent developments in high-throughput sequencing and big-data collection and analysis have driven breakthroughs in genomics, evolutionary biology, and population dynamics. In the field of paleovirology, the recovery and reconstruction of ancient pathogen DNA has pulled back the curtain on the origins and evolution of a number of human-associated pathogens, some of which have been identified as the causative agents of devastating past pandemics. One such pathogen is the Variola virus — a double-stranded DNA virus known to be the causative agent of smallpox — which famously caused huge mortality to numerous human populations in the past, including the Native Americans when introduced by European colonists. The history record of this virus suggests that unmistakable descriptions did not appear until 4AD in China. The results generated by combining this information with novel sequencing methods led some to believe that the virus is likely much older than we thought, with some estimates at around $\sim6000$ years for the divergence between the P1 and P2 strains. \\To investigate this matter further, \cite{muhlemann2020diverse} assembled a data set of Viking Age human gene samples, from which they recovered and authenticated 13 ancient Variola virus genomes; 11 from the Viking Age (600CE-1050CE) and two from the 19th century. Upon phylogenetic analysis, they found that many of the samples form a distinct and previously unknown clade on the Variola virus tree, and that, consistent with previously overlooked findings \citep{duggan201617th}, a common ancestor existed about 1700 years ago, pushing the known genomic history of the virus back over a millenium.\\However, this is still a relatively narrow window of human history. What kind of viruses were affecting early humans? Until recently, a deficit of preserved samples had restricted attempts to look back with any clarity. The Yana Rhinoserous Horn Site — a high arctic site containing a wealth of well-preserved human remains and artifacts — yielded two preserved milk teeth, in which genomic analysis found hints of microbial genomes including Adenovirus, Mastadenovirus C, and Herpes. These reads were mapped to their respective reference genomes to reveal genome-wide coverage for 5 different virus species’ DNA in reconstruction, with highest coverage for Human Adenovirus C. It was found that the Yana viruses are more silimar to modern C1 and C2 viruses than they are to each other, demonstrating all in all that: a) isolation of virus genomes from Pleistocene human remains is possible; b) Viruses causing childhood infections were circulating in human population during the Upper Paleolithic; and c) Human Adenovirus C has slow mutation rate, supporting of long-term co-divergence with host species.


\section{Neural networks for population genetics: demographic inference and data generation}
\textit{Flora Jay\\University of Paris-Sud\\11/02/2021}
\\
\\The advent of new sequencing technologies has driven explosions in unprecedentedly large genomic data sets. Yet extracting information from these sets can be challenging, demanding increasingly sophisticated technology for analysis. Traditionally, techniques/statistics such as Site Frequency Spectrums (SFS) and Approximate Bayesian Computation (ABC) — a framework for likelihood-free inference based on simulations and summary statistics — have been widely used in population genetics for tasks such as demographic inference, but new frontiers in Artificial Intelligence have provided tools that can be leveraged in this domain. Artificial Neural Networks (ANN) are a family of powerful, general-purpose Machine Learning models that are capable of learning highly intricate properties of raw input data without demanding any domain-specific expertise of the user or the computation of a potentially over- or under-supplied set of summary statistics, giving them a marked \emph{prima facie} advantage over the more conventional aforementioned techniques. Building on previous findings that the parameter estimates yielded by ANN constitute suitable inputs to ABC, \cite{sanchez2020deep} constructed a specialised (convolutional and permutation-invariant) deep neural network and trained it on simulated genomic data to draw a comparison between the performances a number of combinations of deep learning and ABC architectures. They found that a combination of their novel architecture (SPIDNA) and ABC provided the best results in minimising the prediction error on simulated data, with good performance in the absence of hand-crafted summary statistics and marginally better with their inclusion, demonstrating (with added force from a more recent result that a pure neural network can improve performance even further) that DNNs provide an excellent framework for optimising demographic inference.


\section{The evolution of human commensalism in Passer sparrows}
\textit{Mark Ravinet\\University of Nottingham\\11/03/2021}
\\
\\Anthropogenic pressures have shaped the evolution of other species for many millennia with varying results. While most are adversely affected, some manage to adapt to, and even thrive in, the anthropocentric modern world. Human commensal species — species that benefit from human produce with no direct intervention from humans themselves — represent a special case of this phenomenon, and have arisen independently within and across genera. To investigate the factors driving the repeated evolution of human commensalism, .. et al. used genomic and phenotypic data from multiple species from the Passer genus to reconstruct the development of commensalism in each species. The tree sparrow was used as an outgroup to test for introgression between spanish sparrow and the commensal populations, where an increase in the D-statistic for all commensal populations was observed, suggesting that these share more alleles in common with Spanish sparrow than what would be expected by chance. SFS analysis revealed a split between Spanish and ancestral house sparrows occurred within last million years, and a split between Bactrianus and House sparrows approximately 11,000 years ago, coincident with the onset of agriculture during the Neolithic revolution, strongly supporting the hypothesis that a relationship has long existed between these commensal species and human populations.

\section{Oil palm and biodiversity: towards more sustainable palm oil production in the tropics}
\textit{Matt Struebig\\Durrell Institute of Conservation & Ecology\\18/03/2021}
\\
\\The agriculture of oil palm—the source of the world's most produced vegetable oil— has long generated controversy, and is thought of by many to be a leading cause of deforestation in the tropics. Enormous international demand for the products of this industry has created a huge amount of revenue and employment opportunities for those living in such countries, and is often promoted as a means of improving living standards in local communities. However, its socio-environmental ramifications are also well known, ranging from land conflict to water and biodiversity loss. The Roundtable on Sustainable Palm Oil (RSPO) — a not-for-profit that mediates negotiations between stakeholders in the palm oil industry for more sustainable production — introduced a sustainability certification system to help mitigate these issues, to which 19\% of the world’s palm oil production is enlisted. However, its effectiveness is still unclear. DICE research in Malaysia and Indonesia investigate ways that biodiversity can be improved within plantation estates. By first identifying which factors drive wildlife persistence in such plantations and then evaluating the occupancy of mammals against each of these covariates, occupancy response curves were partitioned via Bayesian change point analysis and used to identify areas that represent conservation and restoration priorities. A study was set up to determine what levels of wildlife and biodiversity can persist in riparian buffers in the oil palm, and what width these would have to be to sustain specific levels. It was determined that the required width varies a lot between species, but would have to be at least 570-600m wide to support 94\% of all species. Compared to uncertified landscape, certified landscape also had far more connections and more survival in forest patches.

\section{The honeybee waggle dance: evolutionary marvel but modern-day relic?}
\textit{Elli Leadbeater\\Royal Holloway University\\26/11/2020}
\\
\\Foragers of the western honey bee (Apis mellifera) famously exhibit a communicative figure-eight-shaped waggle dance in order to inform their nest mates of food sources, but the importance of this ritual in increasing the food intake of the colony is unclear. It was originally thought that this behaviour was crucial for foraging, however it has since been seen that the obstruction of such communications in temperate environments produces negligible effect on aggregate colony food intake, likely due to simultaneous communication taking place through alternative — perhaps olfactory or pheromonal — channels. This has led many to believe that the phenomenon remains only as a redundant vestigial behaviour, not deleterious enough to be phased out by natural selection but carrying little evolutionary benefit outside of the tropical environment the Apis are known to have evolved within. So how does such a behaviour spread to begin with? Using network-based diffusion analysis, \cite{leadbeater2019honeybee} were able to compare the contributions of multiple information networks to this process by estimating the strength of social transmission. They found that when notifying other bees of a new food source, 93.4\% of transmission events were explained by the waggle dance, with the remaining 6.6\% by indeterminate means. When notifying hivemates of a known food source, on the other hand, only 15.1\% of transmission events were explained by the waggle dance, with the remaining 84.9\% by indeterminate means. Interpreting such insights as indicative of resource abundance for the western honey bee, one can infer which environments are most valuable to them. Using species richness analysis, it was seen that the richness of this species is declining in north-western Europe, likely due to modern agricultural practices, and that bees living in urban areas actually traveled shorter distances to find food and produced nectar of a higher quality than their rural counterparts.

\section{Evolutionary ecology in the time of COVID}
\textit{Will Pierce\\Imperial College\\10/12/2020}
\\
\\In late 2019, a coronavirus spilled over from a bat reservoir into human populations and caused the emergence of SARS-CoV-2 in China, which quickly spread to the rest of the world via international air travel. In order to adequately prepare for coming events, it was imperative that world policy makers understood the virus enough to be able to predict its responses to changes in the environment. It is well known that other coronaviruses exhibit seasonal dynamics, and it was reported early on that COVID-19 responded similarly to environmental variables such as temperature and humidity. To investigate these relationships, Will Pierce’s lab — who previously focused on the evolution of species ecological interactions — refocused their efforts on mapping the relationships between environmental variables and COVID-19 transmission rates. Using transmission data collected in the US, Pearse  \emph{et al.} ran a regression model to determine which variables best predict $R_0$. Through the use of principal component analysis they found that all the variables tested had some predictive power, with population density and temperature being the strongest (a positive correlation was observed between $R_0$ and the former, and a negative correlation between $R_0$ and the latter). It was also seen that the effects of these variables were much weaker in states under lockdown than they were pre-lockdown, providing significant evidence in favour of the utility of lockdown to reduce transmission. Acknowledging some important caveats (namely, the un-generalisability of results to other countries and the idiosyncrasies in case reporting between US states), the study eventually concluded that, in the absence of lockdown measures, transmission rates were likely going to sharply increase as temperatures fall in winter - an important finding for policy makers in countries approaching winter. Human behaviour is an extremely important variable in epidemiology and variance in infection rates between countries, but is notoriously difficult to measure or quantify. Understanding the cultural norms and values underpinning a population's compliance with preventative measures such as hand-washing and social distancing could have the potential to determine the direction of a pandemic, and greatly improve future predictive modelling capabilities.

\section{Contemporary evolution and adaptive divergence in open ocean environments: New insights and applications to fisheries management}
\textit{Nina Overgaard Therkildsen\\Cornell University\\25/02/2021}
\\
\\Anthropogenic pressures such as pollution and overfishing threaten the existence of many marine species. Accordingly, an understanding of the way that organisms adapt to these pressures is crucial in informing conservation strategies. Contrary to the traditional belief that most marine animals would show genetic homogeneity over large areas, we now know that they are capable of highly localised adaptation. Advancements in high-throughput sequencing technology have played an especially salient role in driving such discoveries, enabling unprecedented insights into cases of intra-specific divergence. For a small estuarine fish called the Atlantic silverside, for example, intrinsic growth capacity is strongly correlated with the length of the growing season, itself determined by latitude. While fish sampled from the wild show little phenotypic variation across space, fish in laboratory environments sampled from different latitudes exhibit differing growth rates. This phenomenon of genetic and environmental forces opposing one another is known as countergradient variation, and in the presence of gene flow represents a strong genomic basis for local adaptation along a steep latitudinal gradient. Additionally, the aforementioned advancements in sequencing technology has also enabled insights pertaining to fisheries management that have been overlooked hitherto. Namely, comparative genomic studies of the Atlantic cod have revealed patterns of differentiation incongruent with current management units, and that fisheries conditions can exert strong selective pressures on local populations, pushing rapid evolution.



\section{Understanding societal demands and ecological processes to promote sustainable consumption.}
\textit{Ming Lee\\Sun-Yat Sen University\\17/06/2021}
\\
\\With growing populations demanding evermore food from an increasingly fragile planet, it is essential that humans establish sustainable consumption practises if we are to adequately protect global ecosystems. However, balancing the needs of the Earth and the humans living on it is a difficult task that requires considering issues from a number of angles. For example, oilseed production is cited as a key source of socio-economic wellbeing in many tropical countries, yet it is calculated that pursuing a ‘business-as-usual’ course of production and consumption here will require up to an additional ∼12 million hectares of land by 2040, likely producing catastrophic consequences on an already-strained ecosystem. Another salient example of environmentally unsustainable practice is the illegal wildlife trade. Using data mined from the web, the key nodes in the global wildlife trade network can be identified. From the results it can be seen that China is one of the world's largest contributors to the trade of endangered wildlife. Understanding the cultural substrate for this phenomenon (such as superstitious and customary belief in things like traditional Chinese medicine) is crucial to curbing it, but it is a complex and multifaceted issue. Following the outbreak of COVID-19, the Chinese government imposed a ban on the trade and consumption of wildlife meat. While introduced on the grounds of healthcare, the move was welcomed by animal welfare activists who encouraged further measures to widen the bans. If successful, however, these could produce unintended ramifications for a number of stakeholders, while not necessarily doing much to mitigate any health risks. The Chinese population has a long history of using wildlife products for numerous means, and their production has for some been an occupational way of life for many generations. The activity has also been identified by the country's policy makers as economically lucrative, with very low barriers to entry, and the ban left rural authorities concerned that they could no longer meet economic targets and farmers concerned for their livelihood. Compensation packages were delivered, but it is too early to determine whether these will reach who they need to, or whether more broadly the ban will drive changes in public demand.


\newpage
\bibliography{Biblio.bib}
\end{document}
